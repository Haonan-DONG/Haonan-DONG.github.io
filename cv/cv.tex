\documentclass[a4paper]{article}
\usepackage{fancyhdr}
\usepackage{graphicx}
\usepackage[headsep=0.5cm,headheight = 2cm]{geometry}
\usepackage{setspace}
\usepackage{indentfirst}
\usepackage{enumitem}
\geometry{left=2cm,right=2cm,top=3cm,bottom=1.5cm}
\setstretch{1.45}
\setlength{\parindent}{0pt}


\pagestyle{fancy}
\lhead{\includegraphics[scale = 0.65]{../images/whu_logo.png}}
\rhead{Mr. DONG Haonan\\Tel:+86 - 18966710086\\Email: haonandong@whu.edu.cn\\website: haonan-dong.github.io}
\cfoot{\thepage}
\renewcommand{\headrulewidth}{0.5pt}
\renewcommand{\footrulewidth}{0.5pt}

\begin{document}
\section{Education Background}
\begin{itemize}[itemsep = -0.5em,topsep = 0em]
    \item \textbf{B.S of Geodesy and Geomatics }
          \\\ School of Geodesy and Geomatics.\ Wuhan University\\ Wuhan, China\ \ 2016 - 2020
    \item \textbf{2020} \hspace{3.36cm} Outstanding Graduate Reward
    \item \textbf{2017 - 2019} \hspace{2cm} Second Class Scholarship (Three Times) \hspace{2cm} \textbf{WHU}
    \item \textbf{2017 - 2019} \hspace{2cm} Excellent Student Award (Three Times) \hspace{2cm} \textbf{WHU}
\end{itemize}

\begin{itemize}[itemsep = -0.5em,topsep = 0em]
    \item \textbf{M.S of Pattern Recognition and Intelligent System}
          \\\ School of Remote Sensing and Information Engineering \ Wuhan University\\ Wuhan, China\ \ 2020 - 2023
    \item \textbf{2021} \hspace{3.36cm} Second Class Scholarship \hspace{3.18cm} \textbf{WHU}
\end{itemize}

\section{Publications}
 [1] \textbf{Haonan Dong}, Xiaotong Ye, Congpu Hao. Emergency Evacuation Path Planning Algorithm for Indoor Fire in Commercial Buildings[J]. Journal of Geomatics, 2021,46(S1):40-43.DOI:10.14188/j.2095-6045.2019351.

\section{Research Experience}
% \textbf{07/2019 - Now.\ Key Member.\  Point Cloud Acquisition and Registration}\\
% $Advisor:\ Prof.\ Deng\ Fei$\\
% \textbf{Objective:}\ Use ToF camera to acquire 3D point cloud, then realize registration of point cloud.
% \begin{itemize}[itemsep = -0.5em,topsep = 0em]
%     \item Install Ubuntu System, and learn basic algorithms of point cloud registration and SLAM.
%     \item Use ToF camera to acquire point cloud.
%     \item Point Cloud Registration with MeshLab and PCL algorithm.
% \end{itemize}
% \textbf{This project is still in process, and at present, I mainly deal with calibration of ToF and RGB Camera.}\\

% \textbf{10/2018 - 05/2019.\ Key Member.\ Inertial Navigation on Smartphone-based Platform, WHU}\\
% $Advisor:\ Prof.\ DENG\ Fei$\\
% \textbf{Objective:}\ To Utilize MEMS of mobile phone to make an inertial navigation software and visualize movement trajectory of people. Installed MacOS system and Xcode, tested the adaptation with iPhone.
% \begin{itemize}[itemsep = -0.5em,topsep = 0em]
%     \item Learned Objective-C language, programmed and read the inertial data of MEMS.
%     \item Extracted the original inertial data program, fixed the original data of accelerometer and gyroscope with band-pass filter.
%     \item Used zero-velocity-detection algorithm and pedestrian-based algorithm to calculate single step length.
%     \item Merged step length and orientation to visualize the track of smartphone.
% \end{itemize}
% \textbf{This Project has ended. 2D dead reckoning code will soon be shown on my Website.}\\

\textbf{03/2018 - 06/2019.\ Leader.\ National Students' Platform for Innovation and Entrepreneurship Training Program, WHU}\\
$Advisor:\ Prof.\ YAO\ Yibin,\ Prof.\ HUA\ Xianghong$\\
\textbf{Topic:}\ A Commercial Building Fire Evacuation System Based on an Indoor Path Planning Algorithm and ZigBee Platform.\\
% \textbf{Objective:}\ To design an evacuation system that can provide an escape route for indoor fire through combining ZigBee Wireless Sensor Network.
% \begin{itemize}[itemsep = -0.5em,topsep = 0em]
%     \item Install ZigBee WLAN Web, including Smoke Sensor, Infra-red sensor, and LED note, in the indoor environment.
%     \item Vectorized the map of the interior of the targeted building by using the ArcGIS, adopted the program design of ZigBee to get the numbers of people of each room inside the building and whether the room was the origin of fire.
%     \item Transferred the data to the computer program by wireless sensor network nodes and gorge line, extracted the data in the form of reading and writing files with C\# program.
%     \item Utilized the combined weight method based on AHP weight and entropy weight, calculated the escape order of each node in the room by using the existing data \textbf{(the application of mathematical modeling)}.
%     \item Applied the improved Dijkstra algorithm \textbf{(shortest path algorithm)} to calculate the optimal route, and transmitted the route to the switchboard node of wireless sensor network.
%     \item Lighted up the light source node on demand by switchboard node, which made the light source display the escape direction of corresponding indoor node by depicting an arrow.
% \end{itemize}
\textbf{Achievement}
\begin{itemize}[itemsep = -0.5em,topsep = 0em]
    % \item \textbf{Paper(accepted by Journal of Geomatics):}\ \ \textbf{Dong Haonan},\ Ye Xiaotong, Hao Congpu.\ Emergency Evacuation Path Planning Algorithm for Indoor Fire In Commercial Buildings.\\
    \item \textbf{National Geomatics Technical Essay Contest: Special Prize (The Highest Award)\\
              Geomatics Skill Contest of SGG, Wuhan University: Third Prize}
    \item \textbf{Patent(accepted)}: A Indoor Fire Evacuation Assistant System basing on ZigBee. Wuhan University.
\end{itemize}

\section{Academic Competitions}
\textbf{01/2019:\ MCM/ICM\hspace{2cm}Honorable Prize\hspace{2cm}Leader,\ WHU\\}
\textbf{Topic:} Evacuation Problem in Louvre.\\
% \begin{itemize}[itemsep = -0.5em,topsep = 0em]
%     \item Analyzed prediction method and algorithm, and decided the theme.
%     \item Read large numbers of papers to confirm the algorithm and model: an improved shortest path algorithm based on hierarchical analysis model.
%     \item Used C\# and Matlab to realize and solve the initial model.
%     \item Improved the weight of Dijkstra algorithm to solve the dynamic problem of static algorithm\\
% \end{itemize}

\textbf{08/2019:\ China Undergraduate Mathematical Contest in Modeling\\Third Prize of Hubei Province\hspace{2cm}Key Member,\ WHU\\}
\textbf{Topic:}  Scheduling Problem of Intelligent RGV.
% \begin{itemize}[itemsep = -0.5em,topsep = 0em]
%     \item Analyzed the estimated method and algorithm and determined the topic.
%     \item Discussed with teammates and confirmed our solution which is Monte Carlo method based on stochastic scheduling.
%     \item Used Matlab to program and realize solution of the initial model.
%     \item Utilized SPSS software for statistical analysis of the results, and wrote the thesis.
% \end{itemize}

% \section{Extracurricular Activities}
% \textbf{03/2017 - 03/2019\ Leader \& Volunteer,\ Wanlin Art Museum,\ WHU}
% \begin{itemize}[itemsep = -0.5em,topsep = 0em]
%     \item Mainly responsible for promoting and planning, advertized the museum via platforms like Wechat and Weibo.
%     \item Assigned tasks to students and coordinated teachers of the Museum; designed planning scheme of activities.
%     \item Won the title of Social Activist and Excellent Volunteer respectively (TWICE)\\
% \end{itemize}

% \textbf{09/2016 - 05/2017\ Key Member,\ The Center of Innovation and Practice,\ WHU}
% \begin{itemize}[itemsep = -0.5em,topsep = 0em]
%     \item Designed forms and helped other students to solve affairs.
%     \item Made hardware of media and displayed PPT.
% \end{itemize}

% \section{Skills}
% \subsection{English Proficiency}
% TOEFL 97\ (25 + 25 + 22 + 25) \ \textbf{Will Take Another Test.}\\
% GRE 321 (154 + 167) + 3
% \subsection{Computer Proficiency}
% \textbf{Language:}\ C++,\ Python,\ C\#,Matlab.\\
% \textbf{System:}\ Window,\ Ubuntu

\end{document}